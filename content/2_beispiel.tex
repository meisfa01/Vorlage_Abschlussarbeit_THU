\chapter{Beispiele}
In den nachfolgenden Abschnitten sind einige Codebeispiele aufgezeigt, die zur Verwendung von Latex hilfreich sein können.

\section{Textformatierung}

\subsection{Zeilenumbrüche}
Zeilenumbrüche im Editor erzeugen keine neue Zeile in der PDF. 
Deshalb ist hier kein Zeilenumbruch zu sehen.
Um eine neue Zeile zu beginnen wird die vorherige durch einen doppelten Backslash beendetet.\\
Anschließend wird eine neue Zeile begonnen. 
Um eine Leerzeile zu erzeugen wird die neue Zeile direkt wieder durch einen doppelten Backslash beendet.\\
\\
Der Text geht dann hier weiter.

\subsection{Schriftvarianten}
Um einen Text \textit{kursiv} oder \textbf{fett} zu schreiben werden die Befehle "`textit"' und "`textbf"' verwendet. Damit die korrekte Verwendung von Anführungszeichen sichergestellt werden kann sollte in Textstudio unter Optionen - Textstudio konfigurieren - Editor - Ersetze Anführungszeichen die deutsche Variante ausgewählt werden.\\
\\
Andere Schriftarten - beispielsweise zur Beschreibung von Codeausschnitten - werden durch den Befehl fontfamily aktiviert. Damit diese nur in einem bestimmten Bereich aktiv ist wird der Befehl von geschweiften Klammern umgeben:\\
{\fontfamily{lmtt}\selectfont print("`hello world"')}

\subsection{Akronyme}
Akronyme werden in der Datei "`Abkürzungsverzeichnis.tex"' im Ordner "`framework"' definiert. Bei der Verwendung eines Akronyms wird automatisch das Wort bei der ersten Nennung vollständig ausgeschrieben. Anschließend wird nur noch die Abkürzung genannt:\\
Diese Abschlussarbeit wird an der \ac{thu} verfasst. Die \ac{thu} befindet sich in Ulm.
\\

\section{Umgebungen}

\subsection{Bilder}
Bilder müssen im Ordner "`pictures"' abgespeichert werden, damit diese im Code verwendbar sind.
Die Größe des Bildes wird in Abhängigkeit der Seitenbreite angegeben. Hier sind es 50\% der Seitenbreite. Die Buchstaben [ht] geben an, dass das Bild an dieser Stelle (h = here) platziert werden soll. Falls das nicht möglich ist wird das Bild ganz oben auf der Seite eingefügt (t = top). Durch das Label kann die Abbildung referenziert werden:\\
Auf \autoref{pic:logo_hs} ist das Logo der Hochschule zu sehen.
\begin{figure}[ht]
	\centering
	\includegraphics*[width=0.5\linewidth]{thu_logo}
	\caption{Logo der Hochschule}
	\label{pic:logo_hs}
\end{figure}

\subsection{Stichpunkte}
\begin{itemize}
	\item Stichpunkt 1
	\item Stichpunkt 2
\end{itemize}

\subsection{Mathe}
Um die Matheumgebung zu aktivieren gibt es mehrere Möglichkeiten. Für Formel im Text wird die Formel durch \$ eingeschlossen. Beispielsweise $sin(\pi) = 0$. Für Formeln in einer eigenen Zeile mit zusätzlicher Nummerierung wird die equation Umgebung verwendet.
\begin{equation}
	y = 2 \cdot x
	\label{eqn:formel}
\end{equation}
Diese kann wie bei Bildern durch das Label referenziert werden. Siehe \autoref{eqn:formel}.

\subsection{Diagramme}
Um Diagramme aus Datensätzen zu zeichnen wird das Tikz Paket verwendet. Die Datensätze sollten im csv-Format vorliegen und werden im Ordner "`data"' abgespeichert. 
\begin{figure}[ht]
	\centering
	\begin{tikzpicture}
		\begin{axis}[
			width=10cm, height=5cm,
			axis lines=middle,
			xlabel={t [s]},
			xlabel style={below right},
			ylabel={x [mm]},
			ylabel style={above left},
			ymin=-0, ymax=10,
			xmin=-0, xmax=27,
			xtick={5, 10, 15, 20, 25},
			ytick={2, 4, 6, 8}
			]
			\addplot table [x=time, y=xData, col sep=comma] {data/plotexample.csv};
			\addlegendentry{x-Daten}
		\end{axis}
	\end{tikzpicture}
	\caption{Beispielplot}
	\label{tikz:beispielplot}
\end{figure}


\section{Quellen}
Es ist hilfreich verwendete Quellen durch ein Literaturverwaltungssystem wie "`Citavi"' zu organisieren. In diesem Programm ist es möglich die Quellen als .bib Datei in das Projekt zu exportieren. Allen Quellen wird in "`Citavi"' ein Key zugeordnet. Über diesen werden die Quellen mit dem "`cite"' Befehl aufgerufen \cite{thu}.

\section{ToDos}
Zu erledigende Aufgaben bzw. Kommentare können mit dem "`todo"' Befehl im Text vermerkt werden. Diese werden nur sichtbar, wenn in der preamble die Zeile {\fontfamily{lmtt}\selectfont newcommand(publish)} auskommentiert wird. 
\todo{Hier sollten noch weiter Beispiele eingefügt werden}